\documentclass[article]{jss}
\usepackage[utf8]{inputenc}

\providecommand{\tightlist}{%
  \setlength{\itemsep}{0pt}\setlength{\parskip}{0pt}}

\author{
Derek Corcoran\\University of Missouri \And Elisabeth Webb\\University of Missouri \And Dylan Kesler\\University of Missouri
}
\title{Selecting priority areas from Diversity and individual species abundance
\pkg{DiversityOccupancy}}
\Keywords{keywords, not capitalized, \proglang{Java}}

\Abstract{
The abstract of the article.
}

\Plainauthor{Derek Corcoran, Elisabeth Webb, Dylan Kesler}
\Shorttitle{\pkg{DiversityOccupancy}: Selecting Priority Areas}
\Plainkeywords{keywords, not capitalized, Java}

%% publication information
%% \Volume{50}
%% \Issue{9}
%% \Month{June}
%% \Year{2012}
\Submitdate{}
%% \Acceptdate{2012-06-04}

\Address{
    Derek Corcoran\\
  University of Missouri\\
  First line Second line\\
  E-mail: \href{mailto:corcoranbarriosd@missouri.edu}{\nolinkurl{corcoranbarriosd@missouri.edu}}\\
  URL: \url{http://rstudio.com}\\~\\
      }

\usepackage{amsmath}

\begin{document}

\section{Introduction}\label{introduction}

This template demonstrates some of the basic latex you'll need to know
to create a JSS article.

\subsection{Code formatting}\label{code-formatting}

Don't use markdown, instead use the more precise latex commands:

\begin{itemize}
\itemsep1pt\parskip0pt\parsep0pt
\item
  \proglang{Java}
\item
  \pkg{plyr}
\item
  \code{print("abc")}
\end{itemize}

\section{R code}\label{r-code}

Can be inserted in regular R markdown blocks.

\begin{CodeChunk}
\begin{CodeInput}
x <- 1:10
x
\end{CodeInput}
\begin{CodeOutput}
 [1]  1  2  3  4  5  6  7  8  9 10
\end{CodeOutput}
\end{CodeChunk}

Bibliography

\cite{healy_metabolic_2013} lay the theretical background

Blah blah
\citetext{\citealp[see][pp.~33-35]{healy_metabolic_2013}; \citealp[also][ch.~1]{smith04}}.

\end{document}

